% -------- PACKAGES --------------------------------------------------------
% Document Class
\documentclass[12pt]{report}

% Page Layout & Formatting
\usepackage[a4paper,landscape,width=150mm,top=10mm,bottom=20mm,left=20mm,right=20mm]{geometry} % Layout
\usepackage[many]{tcolorbox} % Boxes (Definitions, Examples)
\usepackage{graphicx} % Images / Other Graphics
\graphicspath{{../graphics/}}
\usepackage{float} % Floats (like Minipage) alignment
\usepackage{parskip} % no indents in new par
\usepackage[hidelinks]{hyperref} % Hyperrefs
\usepackage{bookmark}% Bookmarks

% Math Mode
\usepackage{amsmath}
\usepackage{cancel}

% Language Specific
\usepackage[utf8]{inputenc}
\usepackage[german]{babel} 

% Tables & Arrays
\usepackage{array}   % column types / tables
\usepackage{multicol} % tables, multicolumn
\usepackage{tabularx}

\usepackage{caption} % Caption styling
\usepackage{xcolor}

\newcommand*\sepline{%
   \begin{center}
     \rule[1ex]{\textwidth}{.5pt}
   \end{center}}

% -------- MATH MODE -------------------------------------------------------
% Math Mode Table Columns
\newcolumntype{L}{>{$}l<{$}}
\newcolumntype{C}{>{$}c<{$}}

% Überträge
\def\cred{\hbox to0pt{\textcolor{blue}{${}_{\tt 1}$}\hss}}


% -------- DOCUMENT -------------------------------------------------------
\begin{document}

% ----------- Allgemein & A-/L-Anweisungen -------------------
\begin{samepage}
    \begin{figure}[H]
        \begin{minipage}[t]{0.45\textwidth}
            \begin{center}
                \Huge
                Hack-Assembly - Symbole
            \end{center}
            \sepline

            \begin{table}[H]
                \caption*{Anweisungstypen}
                \begin{tabular*}{\textwidth}{@{\extracolsep{\fill}}|l|ll|}
                    \hline
                    Anweisungstyp & Bedeutung                       &\\ \hline
                    A-Anweisungen & address instructions            &\\
                    C-Anweisungen & compute instructions            &\\
                    L-Anweisungen & labels declaration instructions &\\ \hline
                \end{tabular*}
            \end{table}

            \begin{table}[H]
                \caption*{Vordefinierte Symbole}
                \centering

                \begin{tabular*}{\textwidth}{@{\extracolsep{\fill}}|c|ll|}
                    \hline
                    Symbol & Beschreibung                          &\\\hline
                    A      & A-Register (Addressenregister)        &\\
                    D      & D-Register (Datenregister)            &\\
                    M      & Hauptspeicher-Register (Adresse A)    &\\
                    SP     & RAM-Addresse 0                        &\\
                    LCL    & RAM-Addresse 1                        &\\
                    ARG    & RAM-Addresse 2                        &\\
                    THIS   & RAM-Addresse 3                        &\\
                    THAT   & RAM-Addresse 4                        &\\
                    R0-R15 & RAM-Register (16)                     &\\
                    SCREEN & 16384 Adresse des Bildschirmspeichers &\\
                    KBD    & 24576 Adresse des Tastaturregister    &\\ \hline
                \end{tabular*}
            \end{table}
        \end{minipage}
        \hfill
        \begin{minipage}[t]{0.45\textwidth}
            \begin{center}
                \Huge
                Hack-Assembly - A-Anweisungen
            \end{center}
            \sepline

            \centering
            \begin{tabular}{l}
                \textbf{@value} \\
                \textbf{SOMETHING}
            \end{tabular}

            \begin{table}[H]
                \caption*{Verwendung der A-Anweisung}
                \begin{tabular}{|c|c|l|}
                    \hline
                    $@value$ & Vorgang                   & Beschreibung                 \\\hline
                    $D=A$    & $D \leftarrow value$      & Laden einer Konstante        \\
                    $D=M$    & $D \leftarrow RAM[value]$ & Auswahl Datenspeicherzelle   \\
                    $JMP$    & $fetch\ ROM[value]$       & Auswahl Befehlsspeicherzelle \\ \hline
                \end{tabular}
            \end{table}

            \begin{center}
                \Huge
                Hack-Assembly - L-Anweisungen
            \end{center}
            \sepline

            \centering
            \begin{tabular}{l}
                \textbf{(LABEL)}              \\
                \hspace{10pt} // Instructions \\
                \hspace{10pt}\textbf{@LABEL}  \\
                \hspace{10pt}\textbf{0; JMP}
            \end{tabular}

            Die Anweisung (LABEL) deklariert ein neues Label mit dem Name 'Label'. Der Assembler
            übersetzt diese dann in die Addresse der nächsten Anweisung (nachfolgende Zeile)
        \end{minipage}
    \end{figure}
\end{samepage}


\pagebreak


% ----------- C-Anweisungen -------------------
\begin{samepage}
    \begin{center}
        \Huge
        Hack-Assembly - C-Anweisungen
    \end{center}
    \sepline

    \centering
    \begin{tabular}{ccccc}
        \textbf{DEST}               & = & \textbf{COMP}            & ; & \textbf{JUMP}   \\
        Speicherort des Ergebnisses &   & Auszuführende Berechnung &   & Sprungbedingung
    \end{tabular}
    \begin{figure}[H]
        \begin{minipage}[t]{0.45\textwidth}
            \centering
            \begin{table}[H]
                \caption*{Computation Symboltabelle (ALU)}
                \centering
                \begin{tabular}{llL}
                    x: D & y: A & m=1 \rightarrow A=M \\
                \end{tabular}

                \begin{tabular*}{\textwidth}{@{\extracolsep{\fill}}|CC|CC|CC|C|C|}
                    \hline
                    zx & nx & zy & ny & f & no & m=0      & m=1      \\ \hline
                    1  & 0  & 1  & 0  & 1 & 0  & 0        &          \\
                    1  & 1  & 1  & 1  & 1 & 1  & 1        &          \\
                    1  & 1  & 1  & 0  & 1 & 0  & -1       &          \\
                    0  & 0  & 1  & 1  & 0 & 0  & D        &          \\
                    1  & 1  & 0  & 0  & 0 & 0  & A        & M        \\
                    0  & 0  & 1  & 1  & 0 & 1  & ~D / !D  &          \\
                    1  & 1  & 0  & 0  & 0 & 1  & ~A / !A  & ~M / !M  \\
                    0  & 0  & 1  & 1  & 1 & 1  & -D       &          \\
                    1  & 1  & 0  & 0  & 1 & 1  & -A       & -M       \\
                    0  & 1  & 1  & 1  & 1 & 1  & D+1      &          \\
                    1  & 1  & 0  & 1  & 1 & 1  & A+1      & M+1      \\
                    0  & 0  & 1  & 1  & 1 & 0  & D-1      &          \\
                    1  & 1  & 0  & 0  & 1 & 0  & A-1      &          \\
                    0  & 0  & 0  & 0  & 1 & 0  & D+A      & D+M      \\
                    0  & 1  & 0  & 0  & 1 & 1  & D-A      & D-M      \\
                    0  & 0  & 0  & 1  & 1 & 1  & A-D      & M-D      \\
                    0  & 0  & 0  & 0  & 0 & 0  & D\ \&\ A & D\ \&\ M \\
                    0  & 1  & 0  & 1  & 0 & 1  & D\ |\ A  & D\ |\ M  \\ \hline
                \end{tabular*}
            \end{table}
        \end{minipage}
        \hfill
        \begin{minipage}[t]{0.45\textwidth}
            \centering
            \begin{table}[H]
                \caption*{Destination Symboltabelle}
                \begin{tabular}{lll}
                    M: Memory[A] & D: D-Register & A: A-Register \\
                \end{tabular}

                \begin{tabular*}{\textwidth}{@{\extracolsep{\fill}}|c|CCC|l|}
                    \hline
                    Symbol & d_1 & d_2 & d_3 & Speicherort                         \\ \hline
                    -      & 0   & 0   & 0   & Keine Speicherung                   \\
                    M      & 0   & 0   & 1   & Memory[A]                           \\
                    D      & 0   & 1   & 0   & D Register                          \\
                    MD     & 0   & 1   & 1   & Memory[A] \& D-Register             \\
                    A      & 1   & 0   & 0   & A-Register                          \\
                    AM     & 1   & 0   & 1   & A-Register \& Memory[A]             \\
                    AD     & 1   & 1   & 0   & A-Register und D-Register           \\
                    AMD    & 1   & 1   & 1   & A-Reg., Memory[A] \& D-Reg. \\ \hline
                \end{tabular*}
            \end{table}
            \begin{table}[H]
                \caption*{Jump Symboltabelle}
                \begin{tabular}{lll}
                    E: EQUAL & G: GREATER & L: LOWER \\
                \end{tabular}

                \begin{tabular*}{\textwidth}{@{\extracolsep{\fill}}|c|CCC|l|}
                    \hline
                    Symbol & j_2 & j_1 & j_0 &   Sprunkbedingung          \\ \hline
                    -      & 0   & 0   & 0   & Spring niemals             \\
                    JGT    & 0   & 0   & 1   & Spring falls $out > 0$     \\
                    JEQ    & 0   & 1   & 0   & Spring falls $out=0$       \\
                    JGE    & 0   & 1   & 1   & Spring falls $out\geq 0$   \\
                    JLT    & 1   & 0   & 0   & Spring falls $out < 0$     \\
                    JNE    & 1   & 0   & 1   & Spring falls $out \not= 0$ \\
                    JLE    & 1   & 1   & 0   & Spring falls $out \leq 0$  \\
                    JMP    & 1   & 1   & 1   & Spring immer               \\ \hline
                \end{tabular*}
            \end{table}
        \end{minipage}
    \end{figure}
\end{samepage}


\end{document}