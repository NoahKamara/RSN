% -------- PACKAGES --------------------------------------------------------
% Document Class
\documentclass[12pt]{report}

% Page Layout & Formatting
\usepackage[a4paper,landscape,width=150mm,top=10mm,bottom=20mm,left=20mm,right=20mm]{geometry} % Layout
\usepackage[many]{tcolorbox} % Boxes (Definitions, Examples)
\usepackage{graphicx} % Images / Other Graphics
\graphicspath{{../graphics/}}
\usepackage{float} % Floats (like Minipage) alignment
\usepackage{parskip} % no indents in new par
\usepackage[hidelinks]{hyperref} % Hyperrefs
\usepackage{bookmark}% Bookmarks

% Math Mode
\usepackage{amsmath}
\usepackage{cancel}

% Language Specific
\usepackage[utf8]{inputenc}
\usepackage[german]{babel} 

% Tables & Arrays
\usepackage{array}   % column types / tables
\usepackage{multicol} % tables, multicolumn
\usepackage{tabularx}

\usepackage{caption} % Caption styling
\usepackage{xcolor}

\newcommand*\sepline{%
   \begin{center}
     \rule[1ex]{\textwidth}{.5pt}
   \end{center}}

% -------- MATH MODE -------------------------------------------------------
% Math Mode Table Columns
\newcolumntype{L}{>{$}l<{$}}
\newcolumntype{C}{>{$}c<{$}}

% Überträge
\def\cred{\hbox to0pt{\textcolor{blue}{${}_{\tt 1}$}\hss}}


% -------- DOCUMENT -------------------------------------------------------
\begin{document}

% ----------- Operationen, Programmablaufsteuerung & Speichersegmente -----
\begin{samepage}
    \begin{figure}[H]
        \begin{minipage}[t]{0.45\textwidth}
            \begin{center}
                \Huge
                Variablen-Sichtbarkeit
            \end{center}
            \sepline

            \begin{table}[H]
                \centering
                \begin{tabular*}{\textwidth}{@{\extracolsep{\fill}}|l|l|l|l|}
                    \hline
                    Variablenart        & Deklaration                                                                              & Sichtbarkeit                                                                 \\ \hline
                    statische Variablen & \begin{tabular}[c]{@{}l@{}}Klassen-\\ Deklaration\end{tabular}                           & Innerhalb Klasse                                                             \\
                    Instanzvariablen    & \begin{tabular}[c]{@{}l@{}}Klassen-\\ Deklaration\end{tabular}                           & \begin{tabular}[c]{@{}l@{}}Innerhalb Klasse,\\ außer Funktionen\end{tabular} \\
                    lokale Variablen    & \begin{tabular}[c]{@{}l@{}}Unterprogramm-\\ Deklaration\end{tabular}                     & \begin{tabular}[c]{@{}l@{}}Innerhalb\\ Unterprogramm\end{tabular}            \\
                    Parametervariablen  & \begin{tabular}[c]{@{}l@{}}Unterprogramm-\\ Deklaration\\ (Parameterlisten)\end{tabular} & \begin{tabular}[c]{@{}l@{}}Innerhalb\\ Unterprogramm\end{tabular}            \\ \hline
                \end{tabular*}
            \end{table}

        \end{minipage}
        \hfill
        \begin{minipage}[t]{0.45\textwidth}
            \begin{center}
                \Huge
                VM - Programmablaufsteuerung
            \end{center}
            \sepline

            \begin{table}[H]
                \centering
                \begin{tabular*}{\textwidth}{@{\extracolsep{\fill}}|l|l|}
                    \hline
                    Segment     & Verwendungszweck                                                                                                                                                    \\ \hline
                    $label\ c$   & \begin{tabular}[c]{@{}l@{}}Definiert Marke im Programmtext, \\ z.B. als Sprungziel \\\  \end{tabular}                                                                     \\
                    $goto\ c$    & \begin{tabular}[c]{@{}l@{}}Springt zu einer Marke im Programmtext \\ (unbedingter Sprung) \\\ \end{tabular}                                                              \\
                    $if-goto\ c$ & \begin{tabular}[c]{@{}l@{}}Springt zu einer Marke im Programmtext, \\ wenn das oberste Stapelelement\\ nicht $0$ ist (Element wird von Stapel entfernt\end{tabular} \\ \hline
                \end{tabular*}
            \end{table}
        \end{minipage}
    \end{figure}

    \begin{figure}[H]
        \begin{center}
            \Huge
            VM - Speichersegmente
        \end{center}
        \sepline
        \begin{table}[H]
            \begin{tabular*}{\textwidth}{@{\extracolsep{\fill}}|l|l|l|}
                \hline
                Segment      & Verwendungszweck                          & Kommentare                 \\ \hline
                $argument$   & Speichert Argumente einer Funktion        & \begin{tabular}[c]{@{}l@{}}von VM dynamisch angelegt,\\ wenn Funktion aufgerufen wird\end{tabular}  \\
                $local$      & Speichert lokale Variablen einer Funktion & \begin{tabular}[c]{@{}l@{}}von VM dynamisch angelegt,\\ wenn Funktion aufgerufen wird\end{tabular}  \\
                $static$     & \begin{tabular}[c]{@{}l@{}}Speichert statische Variablen, \\ die von allen Funktionen in einer VM-Datei geteilt werden\end{tabular}                 & \begin{tabular}[c]{@{}l@{}}von VM für jede VM-Datei angelegt,\\ kann von allen Funktionen in der Datei benutzt werden\end{tabular}  \\
                $constant$   & \begin{tabular}[c]{@{}l@{}}Pseudo-Segment;\\ enthält alle Konstanten $0...32767$\end{tabular}                 & \begin{tabular}[c]{@{}l@{}}durch VM emuliert;\\ kann von allen Funkt. des Programms gesehen werden\end{tabular}  \\
                $this$, $that$ & \begin{tabular}[c]{@{}l@{}}Mehrzwecksegmente; \\ können auf verschiedene Bereiche eines Heaps verweisen\end{tabular}                & \begin{tabular}[c]{@{}l@{}}Jede VM-Funkt. kann verwenden,\\ um ausgewählte Bereiche des Heaps zu verändern\end{tabular} \\
                $pointer$    & \begin{tabular}[c]{@{}l@{}}Zwei-Zellen-Segment, \\ enthält Basisadressen der $this$ und $that$ Segmente\end{tabular}                & \begin{tabular}[c]{@{}l@{}}Jede VM-Funkt kann $pointer$ 0 (1) auf Adresse setzen, \\ sodass $this$ ($that$) Segment auf Speicherbereich ausgerichtet wird, \\ der an dieser Adresse anfängt\end{tabular} \\
                $temp$       & \begin{tabular}[c]{@{}l@{}}Festes Acht-Zellen-Segment;\\ enthält temporäre Variablen zur allgemeinen Verwendung\end{tabular}                & \begin{tabular}[c]{@{}l@{}}Jede VM-Funkt. kann verwenden,\\ beliebiger Zweck, wird von allen Funkt. des Programms geteilt\end{tabular} \\ \hline
            \end{tabular*}
        \end{table}
    \end{figure}
\end{samepage}


\pagebreak
% -------- DOCUMENT -------------------------------------------------------
\begin{samepage}
\end{samepage}


\end{document}